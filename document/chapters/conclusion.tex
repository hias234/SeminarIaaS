\chapter{Conclusions and Future Work}
\label{chap:conclusion}

In this article we have given an overview on the cloud type Infrastructure-as-a-Service (IaaS) allowing on-demand provision of compute, storage and networking resources via a software virtualized environment. When deploying applications on the global network, leveraging cloud computing facilities are a key factor for many successful world-wide operating business and IaaS provides the fundament where platforms and software products can be built on. A huge variety of features, simple configurability and several big internet companies as providers make IaaS a solid solution and attractive for any type of internet business. The flexibility and elasticity of compute resources with their on-demand and pay-what-you-need strategy make it easy to adapt to the current situation and workload and removes some annoying limitations of traditional physical dedicated hardware solutions. Due to software-based management and simple web interfaces as popular IaaS providers like Amazon EC2 or Google Compute Engine (GCE) provide it, the times of setting-up a dedicated server using vim or nano to manually change some SSH options are history. By just a few clicks new infrastructure instances or data centres are ready and running letting companies focus on their business related tasks and nothing else. 

IaaS has moved away from a hyped trend to a well established cloud technology and is sill under ambitious development. Popular providers like AWS or GCE are running big projects and try to further improve simplicity of configuration tasks, high performance infrastructure clusters and availability.

