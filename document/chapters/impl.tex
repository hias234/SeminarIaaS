\chapter{Technical Details}
\label{chap:implementation}

\section{Virtualization}
The most important concept of cloud computing and therefore also for IaaS systems is virtualization. Nowadays every cloud computing system is virtualized. This means that all the user's cloud computing applications run in virtual machines. In the case of IaaS systems the user buys a VM and can install whatever system he likes on it \cite{Arzuaga_2010}. 

Virtualization brings many advantages. A lot of crucial concepts of cloud computing are a lot easier with virtualized systems. On example is load balancing: Virtual machines can be copied and can be migrated on other physical machines without any problems. See Section \ref{sec:load_balancing} \cite{Arzuaga_2010}.

\section{Load Balancing}
\label{sec:load_balancing}
As in the previous section mentioned, the applications on cloud computing systems run in virtual machines, and these VMs run in physical machines. Furthermore, there is a huge variation on the load (or resource needs) on the applications, therefore if there run too many applications (or VMs) on one physical machine it may get overloaded. This is why one needs load balancing. Load balancing should avoid that the physical machines have to handle more resources than they can offer. These resources can be CPU, RAM or storage space. If there run too many VMs on one physical machine this can mean a tremendous slow down of the VMs running on it. But slow VMs and applications is not the only problems, that bad load balancing would cause. If a application does not have sufficient resources, often the Service Level Agreement (SLA) is violated. A SLA is the agreement between customer and provider that guarantees certain parameters like performance and availability. A violation can mean that the cloud computing provider may have to give discount because of the violation or the customer decides to change to another cloud computing provider. As you can see good load balancing is crucial in the field of cloud computing \cite{Chen_2014}. 



\section{Resilience Planning}

\section{Backup Strategies}

\section{Monitoring}