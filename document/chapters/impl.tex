\chapter{Technical Details}
\label{chap:implementation}

\section{Virtualization}
The most important concept of cloud computing and therefore also for IaaS systems is virtualization. Nowadays every cloud computing system is virtualized. This means that all the user's cloud computing applications run in virtual machines. In the case of IaaS systems the user buys a VM and can install whatever system he likes on it \cite{Arzuaga_2010}. 

Virtualization brings many advantages. A lot of crucial concepts of cloud computing are a lot easier with virtualized systems. On example is load balancing: Virtual machines can be copied and can be migrated on other physical machines without any problems. See Section \ref{sec:load_balancing} \cite{Arzuaga_2010}.

\section{Load Balancing}
\label{sec:load_balancing}
As mentioned in the previous section, the applications on cloud computing systems run in virtual machines, and these VMs run in physical machines. Furthermore, there is a huge variation on the load (or resource needs) on the applications, therefore if there run too many applications (or VMs) on one physical machine it may get overloaded. This is why one needs load balancing. Load balancing should avoid that the physical machines have to handle more resources than they can offer. These resources can be CPU, RAM or storage space. If there run too many VMs on one physical machine this can mean a tremendous slow down of the VMs running on it. But slow VMs and applications are not the only problems, that bad load balancing would cause. If a application does not have sufficient resources, often the Service Level Agreement (SLA) is violated. A SLA is the agreement between customer and provider that guarantees certain parameters like performance and availability. A violation can mean that the cloud computing provider may have to give discount because of the violation or the customer decides to change to another cloud computing provider. As you can see good load balancing is crucial in the field of cloud computing \cite{Chen_2014}. 

\subsection{Load Balancing Algorithms}
There are many approaches on how load balancing algorithms should work, but they all have something in common: the input and the output. Every algorithm has to watch over the states of the physical and virtual machines and has to decide if a VM gets moved from. Furthermore, it has to decide which VM has to be migrated to which physical machine \cite{Chen_2014}.

Many common load balancing algorithms are based on the current states of the PMs and VMs. They therefore work reactively. This means that when the resource utilization on a certain PM reaches a certain threshold, a VM gets migrated to another PM. This algorithm is rather easy to implement, as it only has to watch the current resource utilization at the physical machines, but the disadvantages are that it only considers the current state of the system and that when a the threshold is reached most often, an imbalance situation is yet the case. Furthermore, it cannot guarantee a long-term balance situation, as it only acts on the parameters known at that point of time \cite{Arzuaga_2010,Chen_2014}.

Other algorithms are based on a "proactive" approach. In these algorithms the physical machines try to predict the resource demand of the VMs running on them and if in the near future there would be a overload they migrate a VM to another physical machine. This has the advantage that if the algorithm works as desired and the estimates on the resource demands of the VMs are approximately correct, there won't be any more overloads, because the algorithm would predict correct and migrate the VMs before the physical machine is overloaded \cite{Beloglazov_2013,Chen_2014}.

\section{Resilience Planning}

\section{Backup Strategies}

\section{Monitoring}